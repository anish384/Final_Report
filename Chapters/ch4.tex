\chapter{System Requirements}

\section{Functional Requirements}

The enhanced autonomous vehicle perception system must satisfy comprehensive functional requirements that ensure reliable operation across diverse environmental conditions and use cases.

\begin{itemize}
    \item \textbf{Real-Time Object Detection:} The system shall detect and classify vehicles, pedestrians, cyclists, traffic signs, and road obstacles with a minimum of 95\% accuracy under optimal conditions and at least 90\% under adverse weather conditions. Detection latency shall not exceed 100 milliseconds from image capture to result output.
    
    \item \textbf{Robust Lane Detection:} The system shall identify lane boundaries, lane departure scenarios, and provide lane-keeping guidance with 98\% accuracy on roads with clear markings and at least 95\% accuracy on roads with faded or partially occluded markings.
    
    \item \textbf{Weather Adaptation:} The system shall automatically detect environmental conditions including rain, fog, snow, and extreme lighting scenarios, adjusting processing parameters to maintain optimal performance. Weather classification shall complete within 10 milliseconds.
    
    \item \textbf{Temporal Consistency:} The system shall maintain consistent object tracking across video frames, reducing detection flickering by a minimum of 80\% compared to frame-independent processing approaches.
    
    \item \textbf{Multi-Resolution Processing:} The system shall process input video at multiple resolutions simultaneously, enabling both detailed analysis for lane detection and efficient broad-area monitoring for object detection.
\end{itemize}

\section{Performance Requirements}

\begin{itemize}
    \item \textbf{Processing Speed:} The complete perception pipeline shall process 1920$\times$1080 video streams at a minimum of 30 frames per second, with capability for 60 fps on high-performance platforms.
    
    \item \textbf{Memory Usage:} Total system memory utilization shall not exceed 4GB RAM to ensure compatibility with standard automotive computing platforms.
    
    \item \textbf{Power Consumption:} Average power consumption shall not exceed 25 watts during normal operation, in compliance with automotive electrical system constraints.
    
    \item \textbf{Accuracy Metrics:}
    \begin{itemize}
        \item Object detection mean Average Precision (mAP) $\geq$ 0.90 under optimal conditions
        \item Lane detection accuracy $\geq$ 98\% on standard road markings
        \item Weather classification accuracy $\geq$ 95\% across defined weather categories
        \item False positive rate $<$ 2\% for critical safety objects (vehicles, pedestrians)
    \end{itemize}
\end{itemize}

\section{Hardware Requirements}

\begin{itemize}
    \item \textbf{Processing Unit:} Multi-core CPU with minimum 3.0 GHz clock speed; dedicated GPU with minimum 4GB VRAM supporting CUDA or OpenCL acceleration.
    
    \item \textbf{Memory:} Minimum 8GB system RAM, preferably 16GB for optimal performance with concurrent processing threads.
    
    \item \textbf{Storage:} Minimum 100GB available storage for model weights, temporary processing files, and diagnostic logging.
    
    \item \textbf{Camera Interface:} Support for automotive-grade cameras with minimum 1920$\times$1080 resolution at 30fps, preferably with HDR capability for extreme lighting conditions.
    
    \item \textbf{Environmental Specifications:} Operating temperature range of --20°C to +70°C; humidity tolerance up to 95\% non-condensing; vibration resistance in accordance with automotive standards.
\end{itemize}

\section{Software Requirements}

\begin{itemize}
    \item \textbf{Operating System:} Linux-based real-time operating system optimized for automotive applications with deterministic processing scheduling support.
    
    \item \textbf{Deep Learning Framework:} TensorRT or ONNX Runtime for optimized neural network inference, with fallback support for PyTorch or TensorFlow.
    
    \item \textbf{Computer Vision Libraries:} OpenCV 4.5+ for image preprocessing and traditional computer vision operations, with GPU acceleration support.
    
    \item \textbf{Development Environment:} C++ for performance-critical components; Python for prototyping and non-critical processing modules.
\end{itemize}

\section{Safety and Reliability Requirements}

\begin{itemize}
    \item \textbf{Fault Tolerance:} The system shall detect and recover from processing failures within 200 milliseconds, maintaining safe operation through degraded mode functionality.
    
    \item \textbf{Diagnostic Monitoring:} Continuous self-monitoring shall detect performance degradation, hardware failures, and environmental changes that may affect system reliability.
    
    \item \textbf{Fail-Safe Operation:} In the event of critical component failure, the system shall enable safe fallback modes, including simplified object detection and basic lane guidance sufficient for driver handover.
    
    \item \textbf{Data Logging:} Comprehensive logging of detection events, performance metrics, and error conditions shall be maintained for post-incident analysis and system improvement.
\end{itemize}

\section{Security Requirements}

\begin{itemize}
    \item \textbf{Data Protection:} All video data and processing results shall be encrypted during storage and transmission to prevent unauthorized access to vehicle sensor information.
    
    \item \textbf{Software Integrity:} The system shall implement secure boot procedures and runtime integrity checking to prevent execution of malicious code.
    
    \item \textbf{Communication Security:} All external communications shall use authenticated and encrypted protocols to prevent spoofing or data manipulation attacks.
\end{itemize}

\section{Usability and Maintenance Requirements}

\begin{itemize}
    \item \textbf{Configuration Management:} The system shall provide standardized configuration interfaces for parameter adjustment and performance tuning without requiring specialized expertise.
    
    \item \textbf{Update Mechanism:} Support for over-the-air updates of model weights and software components, with rollback capability in the event of update failure.
    
    \item \textbf{Diagnostic Interface:} The system shall provide comprehensive diagnostic and monitoring tools for technicians to assess system health, performance trends, and maintenance requirements.
    
    \item \textbf{Documentation:} Complete technical documentation shall be provided, including installation procedures, configuration guides, troubleshooting protocols, and performance tuning recommendations.
\end{itemize}

